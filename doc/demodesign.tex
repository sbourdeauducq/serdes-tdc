\documentclass[a4paper,11pt]{article}
\usepackage{fullpage}
\usepackage[latin1]{inputenc}
\usepackage[T1]{fontenc}
\usepackage[normalem]{ulem}
\usepackage[english]{babel}
\usepackage{listings,babel}
\usepackage{palatino}
\usepackage{gensymb}
\lstset{breaklines=true,basicstyle=\ttfamily}
\usepackage{graphicx}
\usepackage{moreverb}
\usepackage{url}
\usepackage{tabularx}
\usepackage{float}
\usepackage{tweaklist}
\renewcommand{\itemhook}{\setlength{\topsep}{0pt}\setlength{\itemsep}{0pt}}
\renewcommand{\enumhook}{\setlength{\topsep}{0pt}\setlength{\itemsep}{0pt}}

\title{SERDES-based TDC Core demonstration design}
\author{S\'ebastien Bourdeauducq}
\date{July 2012}
\begin{document}
\setlength{\parindent}{0pt}
\setlength{\parskip}{5pt}
\maketitle{}
\section{Hardware}
The demonstration design runs on a SPEC board equipped with a FMC DIO 5-channel daughterboard.

Data is transferred at 115200 8-N-1 via the serial console accessible using the built-in USB adapter. The PCIe interface is not used.

The system clock is 125MHz and the multiplied clock is 1GHz. Therefore, timestamps are directly in nanoseconds.

Test signals go through the FMC daughterboard. The first LEMO connector on the daughterboard is configured as output and transmits an oscillating pattern. The next two LEMO connectors are inputs connected to TDC channels.

\section{Contents}
The demonstration design contains:
\begin{itemize}
\item the LatticeMico32 soft processor, running at 125MHz from the clock generated by the CDCM61004 chip on the SPEC.
\item UART, timer and GPIO cores from the Milkymist SoC. Memory-mapped.
\item two TDC cores (for dual-channel operation). Memory-mapped using the Wishbone interface.
\item an oscillator for generating TDC test signals.
\item a basic boot ROM and command line interface based on the Milkymist BIOS.
\item software routines for testing the TDC core.
\end{itemize}

\section{Synthesizing and running the design}
\subsection{Compiling the LM32 software}
Since the LM32 software is used in block RAM initialization, it must be compiled before the FPGA bitstream is built. The scripts expect the \verb!lm32-elf! toolchain, but any LM32 toolchain should be suitable.

You will need first to compile some host tools which are used to build block RAM initialization files:
\begin{verbatim}
cd $TDCDIR/demo/tools
make
\end{verbatim}

Then, compile the base library and the software image:
\begin{verbatim}
cd $TDCDIR/demo/software/demo
make
\end{verbatim}

\subsection{Compiling the FPGA bitstream}
The design can be synthesized with ISE 14.1. The compilation is automatic:
\begin{verbatim}
cd $TDCDIR/demo/boards/spec/synthesis
make -f Makefile.xst
\end{verbatim}

\subsection{Downloading the FPGA bitstream}
If you are using Xilinx iMPACT, there is an automatic script:
\begin{verbatim}
cd $TDCDIR/demo/boards/spec/synthesis
make -f Makefile.xst load
\end{verbatim}

Otherwise, load the file \verb!$TDCDIR/demo/boards/spec/synthesis/build/system.bit! with your favorite FPGA programming tool.

\section{Command line interface}
Once the bitstream is loaded, it implements a command-line interface on the serial console (115200 8-N-1), and it displays a \textbf{TDC>} prompt. From there, you can enter the following commands:

\begin{tabularx}{\textwidth}{|l|X|}
\hline
mr <address> [length] & Reads memory. Length is in bytes (default 1). \\
\hline
mw <address> <value> [count] & Writes one or several 32-bit words to memory. Count is in words (default 1). \\
\hline
temp & Displays the current temperature in \degree C measured by the 1-wire sensor. \\
\hline
daclevel <value> & Sets the output voltage on all channels of the I2C DAC5578 digital to analog converter on the FMC DIO board. The 16-bit value is directly written into the DAC. \\
\hline
tdc & Waits for TDC events on both channels and displays (respectively) the channel, the polarity, and the timestamp in CSV format. Sending any character to the console stops the series of measurements. \\
\hline
diff & In a loop, waits for a TDC event to happen in both channels, and displays the difference in the timestamps. The output is in CSV format and contains respectively the polarity of the transition and the time difference between the two channels. Sending any character to the console stops the series of measurements. \\
\hline
\end{tabularx}

\end{document}
